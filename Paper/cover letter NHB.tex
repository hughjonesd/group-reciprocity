\documentclass[a4paper,11pt,leqno]{article}
\usepackage{amsmath,amssymb,amsthm}
\usepackage[semicolon,sort]{natbib}
\usepackage{booktabs,enumitem}
\usepackage{graphicx}
\usepackage{setspace}


\newcommand{\signature}[1]{%
 \par\parbox[t]{0.3\textwidth}\quad#1\par}


\newtheorem{res}{Result}

\onehalfspacing

\begin{document}

\begin{flushright}
  {\today}
\end{flushright}

\vspace{1cm}

\noindent To the editors of \emph{Nature Human Behaviour}
\bigskip


\noindent Dear editors,
\medskip

\noindent 

We hereby submit the manuscript ``Humans reciprocate intentional harm by discriminating against group peers'' for conisderation for publication in \emph{Nature Human Behaviour}.
Cycles of intergroup revenge appear in large scale conflicts. We experimentally test the hypothesis that humans practice group-based reciprocity: if someone harms or helps them, they harm or help other members of that person's group. Our results reveal an interesting distinction. Group reciprocity only emerges when the perpetrator's act is an unequivocal malicious norm violation. Acts that are harmful, but are justifiable or acceptable, lead to direct reciprocity towards the perpetrator, but have no effect on behaviour towards her group peers. We look forward to hearing from you.


\bigskip

\signature{Sincerely,}
\bigskip
\signature{David Hugh-Jones, Itay Ron and Ro'i Zultan}

\end{document}