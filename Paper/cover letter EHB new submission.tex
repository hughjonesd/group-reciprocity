\documentclass[a4paper,11pt,leqno]{article}
\usepackage{amsmath,amssymb,amsthm}
\usepackage[semicolon,sort]{natbib}
\usepackage{booktabs,enumitem}
\usepackage{graphicx}
\usepackage{setspace}


\newcommand{\signature}[1]{%
 \par\parbox[t]{0.3\textwidth}\quad#1\par}


\newtheorem{res}{Result}

\onehalfspacing

\begin{document}

\begin{flushright}
  {\today}
\end{flushright}

\vspace{1cm}

\noindent Prof. Daniel Hruschka \\ Editor \\ \emph{Evolution and Human Behavior}
\bigskip


\noindent Dear Professor Hruschka,
\medskip

\noindent 

We hereby submit the manuscript ``Humans reciprocate intentional harm by discriminating against group peers'' for consideration for publication in \emph{Evolution and Human Behavior}.
This is a new version of our previous submission EVOLHUMBEHAV\_2017\_26.

In line with your suggestions, we have made substantial changes. We have rewritten the introduction. It now focuses 
on the place of group reciprocity in the evolution of intergroup conflict. We discuss the 
`chimpanzee model' of Wrangham and Glowacki (2012) and the rival parochial altruism view, and argue that group reciprocity can help to explain the evidence. We also discuss explicitly how group reciprocity could evolve. In the supplementary
materials, we report evolutionary simulations that confirm the rationale that we put forward in the text.
While the focus of our contribution remains on the experimental results, we believe that the new version helps clarify the role of the proximate mechanism studied in the experiment within the evolution of human cooperation and conflict. We look forward to hearing from you.

% First cover letter
%Sometimes humans take revenge, not on the person who harmed them, but on other people from that person's group. This can lead to intergroup conflict and violence. We ran a laboratory experiment showing that this happens. We found that humans only take revenge on groups when the original person's act was deliberate and unequivocal. Our results provide the first clean evidence for group based upstream reciprocity, and points at the boundaries of the phenomenon. We believe that group based upstream reciprocity is an important phenomenon underlying human cooperation and conflict, and is of interest to the readership of \emph{Evolution and Human Behavior}. We look forward to hearing from you.


\bigskip

\signature{Sincerely,}
\bigskip
\signature{David Hugh-Jones, Itay Ron and Ro'i Zultan}

\end{document}